%%%%%%%%%%%%%%%%%%%%%%%%%%%%%%%%%%%%%%%%%%%%%%%%%%%%%%%%%%%
%%%
%%%            日本伝熱学会主催「日本伝熱シンポジウム」
%%%    【非公式】講演アブストラクト原稿(A4 用紙 1 枚,日本語)
%%%
%%%     <https://github.com/Yuki-MATSUKAWA/NHTS_template>
%%%
%%%                      v1.0.0 Yuki MATSUKAWA xx Apr. 2024
%%%
%%%%%%%%%%%%%%%%%%%%%%%%%%%%%%%%%%%%%%%%%%%%%%%%%%%%%%%%%%%

\documentclass[
    paper=a4paper,      % A4 用紙サイズ
    article,            % article 相当の文書クラス
    fleqn,              % 数式を左寄せ
    fontsize=10pt,      % 欧文サイズ 10 pt
    jafontsize=10pt,    % 和文サイズ 10 pt
    head_space=20mm,    % 天の余白
    foot_space=25mm,    % 地の余白
    gutter=20mm,        % のどの余白
    fore-edge=20mm      % 小口の余白
    ]{jlreq}            % jlreq クラスを使用

%%% 使用するスタイルファイル %%%
\usepackage{settings}

%%% フッターの情報 %%%
\newcommand{\foot}{
  %%% 開催する回に応じて内容を忘れずに変更 %%%
  第xx回日本伝熱シンポジウム講演論文集 (20xx-5)
  %%%%%%%%%%%%%%%%%%%%%%%%%%%%%%%%%%%%%%%%%
  }

\begin{document}

\thispagestyle{nhts}

%%% 著者情報 %%%
\begin{center}
    \fontsize{14pt}{18pt}\selectfont
    {\gtfamily
    ここには和文表題を書いてください.
    }\\\fontsize{12pt}{14pt}\selectfont
    {\gtfamily
    ここには和文副題を書いてください.
    }\\
    Please write the English title here.
    \\
    Please write the English subtitle here.
    \vskip12pt\fontsize{10pt}{12pt}\selectfont
    \begin{tabular}{lrll}
      伝学  &$^\ast$\hspace{-3mm} &伝熱 太郎  &(伝熱大学)\\
        & &伝熱 花子  &(伝熱大学)\\
      伝正  & &抜山 四郎  &(東京伝熱大学)
    \end{tabular}
    \vskip12pt
    Taro DENNETSU$^1$, Hanako DENNETSU$^1$ and Shiro NUKIYAMA$^2$
    \\
    $^1$Dept. of Mech. Eng., Dennetsu Univ., 5-1-5, Kashiwanoha, Kashiwa, 277-8563 \\
    $^2$Dept. of Mech. Eng., Tokyo Dennetsu Univ., 1-18-11, Uchikanda, Chiyoda, 101-0047
    \vskip12pt
    \textit{Key Words}: 
    %%% ここにはこの研究のキーワードを書いてください %%%
    Heat Transfer, Boiling, Two-phase flow, Forced convection, Natural convection
    %%%%%%%%%%%%%%%%%%%%%%%%%%%%%%%%%%%%%%%%%%%%%%%%
    \vskip12pt
\end{center}

\jalipsum[1-4]{wagahai}

\end{document}
